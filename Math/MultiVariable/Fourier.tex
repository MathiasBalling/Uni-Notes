\section{Fourier}
\subsection{Fourier Series}
A periodic function with period $2L$ and let $f(x)$ and $f'(x)$ be piecewise continuous on the interval $-L < x < L$
$$S_{N}(x)=\frac{a_{0}}{2}+\sum_{n=1}^{\infty}\left( a_{n}\cos\left( \frac{n\pi x}{L} \right)+b_{n}\sin\left( \frac{n\pi x}{L} \right) \right)=\sum^{\infty}_{n=-\infty}c_{n}e^{ jn\pi x/L }$$
The coefficients are given by:
$$a_n  =\frac{1}{L} \int_{-L}^L f(x) \cos \left(\frac{n \pi x}{L}\right) d x\qquad   n \geq 0$$
$$b_n  =\frac{1}{L} \int_{-L}^L f(x) \sin \left(\frac{n \pi x}{L}\right) d x\qquad   n>0$$
$$c_{n}=\frac{1}{2}(a_{n}-jb_{n})\qquad \qquad \qquad\qquad n>0$$

\subsection{Fourier Transform}
If $h(t)$ is a periodic function then the Fourier transform is given by:
$$H(\omega)=\int_{-\infty}^{\infty}h(t)e^{ -j\omega t }  \, dt $$
Inverse Fourier tranformation of $H(\omega)$:
$$h(t)=\frac{1}{2\pi}\int_{-\infty}^{\infty} H(\omega)e^{ j\omega t } \, d\omega $$


\begin{table}[h]
\centering
\begin{tabular}{|c|c|}
\hline
\cellcolor[HTML]{C0C0C0} \textbf{Signal}& \cellcolor[HTML]{C0C0C0}\textbf{Fourier Transform}  \\ \hline
$\delta(t)$& 1\\ \hline
$u(t)$& $\frac{1}{j\omega}+\pi\delta(\omega)$ \\ \hline
$\delta(t-t_0)$& $e^{-j\omega t_0}$  \\ \hline
$\sin(\omega_0t)$&$-j\pi(\delta(\omega-\omega_{0})-\delta(\omega+\omega_{0}))$ \\ \hline
$\cos(\omega_0t)$&$\pi(\delta(\omega-\omega_{0})+\delta(\omega+\omega_{0}))$ \\ \hline
1&$2\pi\delta(\omega)$ \\ \hline
\end{tabular}
\end{table}

\subsection{Examples}
\subsubsection{Example 1: Fourier Series}
Find the Fourier coefficients and Fourier Series for the square wave shown below:
$$f(x)=\begin{cases}
  0 &  \text{for }-1\leq x\leq0\\
  1 & \text{for }0\leq x\leq 1
\end{cases}$$
and
$$f(x+2)=f(x)$$


\rule{\textwidth}{0.5pt}

The fourier series is given by:
$$f(x)=\frac{a_{0}}{2}+\sum_{n=1}^{\infty}\left( a_{n}\cos\left( \frac{n\pi x}{L} \right)+b_{n}\sin\left( \frac{n\pi x}{L} \right) \right)$$
Find the $L$ value:
$$2L=2\qquad \Rightarrow \qquad L=1$$

Find $a_0$:
$$a_n  =\frac{1}{L} \int_{-L}^L f(x) \cos \left(\frac{n \pi x}{L}\right) d x\qquad   n \geq 0$$
$$a_0  =\frac{1}{1} \int_{-1}^1 f(x) \cos \left(\frac{0 \pi x}{1}\right) d x=\int_{-1}^1 f(x) d x=\int_{-1}^0 0 d x+\int_{0}^1 1 d x=1$$

Find $a_n$:
$$a_n  =\frac{1}{L} \int_{-L}^L f(x) \cos \left(\frac{n \pi x}{L}\right) d x\qquad   n \geq 0$$
$$=\frac{1}{1} \int_{-1}^1 f(x) \cos \left(\frac{n \pi x}{1}\right) d x=\int_{-1}^1 f(x) \cos \left(n \pi x\right) d x$$
$$=\int_{-1}^0 0 \cos \left(n \pi x\right) d x+\int_{0}^1 1 \cos \left(n \pi x\right) d x=0+\left[\frac{\sin(n\pi x)}{n\pi}\right]^1_0=\frac{\sin (\pi  n)}{\pi  n}$$
For all $n$:
$$\frac{\sin (\pi  n)}{\pi  n}=0$$

Find $b_n$:
$$b_n  =\frac{1}{L} \int_{-L}^L f(x) \sin \left(\frac{n \pi x}{L}\right) d x\qquad   n>0$$
$$b_n  =\int_{-1}^1 f(x) \sin \left(\frac{n \pi x}{1}\right) d x=\int_{-1}^0 0 \sin \left(n \pi x\right) d x+\int_{0}^1 1 \sin \left(n \pi x\right) d x$$
$$=0+\left[\frac{-\cos(n\pi x)}{n\pi}\right]^1_0=\frac{-\cos(n\pi 1)}{n\pi}-\frac{-\cos(n\pi 0)}{n\pi}$$
If $n$ is even the function will cancel out, therefore $n=1,3,5,\dots$ (odd):
$$=\frac{1}{n\pi}+\frac{1}{n\pi}=\frac{2}{n\pi}$$
Ans:
$$f(x)=\frac{1}{2}+\sum_{n=1,3,5,\dots}\frac{2}{n\pi}\sin(n\pi x)$$
\subsubsection{Example 2: Fourier Transform}
The unit step function is defined as:
$$u(t-a)=\begin{cases}
  1&  \text{for }t-a>0\\
  0&  \text{for }t-a<0\\
\end{cases}$$
is used to define the rectangular pulse function:
$$x(t)=u(t-a)-u(t-b)\qquad \text{where }a<b$$

\rule{\textwidth}{0.5pt}
$$X(\omega)=\int_{-\infty}^{\infty}x(t)e^{ -j\omega t }  \, dt $$
$$X(\omega)=\int_{-\infty}^{a}0e^{ -j\omega t }  \, dt +\int_{a}^{b}1e^{ -j\omega t }  \, dt +\int_{b}^{\infty}0e^{ -j\omega t }  \, dt $$
$$X(\omega)=0+\left[\frac{-e^{-j\omega t}}{j\omega} \right]_a^b+0$$

Insert the limits:
$$X(\omega)=\frac{e^{-j\omega a}-e^{-j\omega b}}{j\omega}$$
% \subsubsection{Example 3: Inverse Fourier Transform}
% Consider the signal: $X(\omega)=\delta(\omega-\omega_0)$ where $\omega_{0}$ is a constant. Find the inverse Fourier Transform of this signal to find $x(t)$.
%
%
% \subsubsection{Example 4: Fourier Transform}
% Consider the signal: $x(\omega)=\delta(\omega-\omega_0)$ where $\omega_{0}$ is a constant. Find the inverse Fourier Transform of this signal to find $x(t)$.
% Ans:
% $$X(\omega)=-j\pi(\delta(\omega-\omega_0)-\delta(\omega+\omega_0))$$
