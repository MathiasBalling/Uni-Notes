\section{Fields-Curve}
\subsection{Curve \& Parameterization}
Representation of a curve in 3 space by using its position vector is given as:
$$r=r(t)=x(t)\mathbf{i}+y(t)\mathbf{j}+z(t)\mathbf{k}\qquad{\mathrm{~where~}}a\leq t\leq b$$

\subsection{Vector Fields}
$$\mathbf{F}(x,y,z)=\underbrace{ f_{1}(x,y,z) }_{ \text{Scaler function} }\mathbf{i}
+\underbrace{ f_{2}(x,y,z) }_{ \text{Scaler function} }\mathbf{j}
+\underbrace{ f_{3}(x,y,z) }_{ \text{Scaler function} }\mathbf{k}$$

\begin{eqnarray*}
    \frac{\partial f}{\partial x}=f_1=f_x\qquad
  &\frac{\partial f}{\partial y}=f_2=f_y\qquad
  &\frac{\partial f}{\partial z}=f_3=f_z
\end{eqnarray*}
Position vector:
$$r=x\mathbf{i}+y\mathbf{j}+z\mathbf{k}$$
Unit vector with magnitude 1:
$$r=\mathbf{i}+\mathbf{j}+\mathbf{k}$$
\subsubsection{Scalar field}
$$F(x,y,z)=f_1(x,y,z)+f_2(x,y,z)+f_3(x,y,z)$$
The gradient of a scalar field is a vector field:
$$\nabla f=\text{grad } f(x,y,z)=f_x(x,y,z)\mathbf{i}+f_y(x,y,z)\mathbf{j}+f_z(x,y,z)\mathbf{k}$$

\subsubsection{Field lines}
$${\frac{d x}{f_{1}(x,y,z)}}={\frac{d y}{f_{2}(x,y,z)}}={\frac{d z}{f_{3}(x,y,z)}}$$
\subsubsection{Convervation field}
If $\mathbf{F}(x,y,z)=\nabla\phi(x,y,z)$ in a 3d domain $D$, then $\mathbf{F}$ is a conservative vector field in $D$ and function $\phi$ is the potential function.
$$\mathbf{F}(x,y,z)=\nabla\phi(x,y,z)=\phi_x(x,y,z)\mathbf{i}+\phi_y(x,y,z)\mathbf{j}+\phi_z(x,y,z)\mathbf{k}$$
If the vetor field is conservative, then all the following equations are true:
\begin{eqnarray*}
  \frac{\partial f_1}{\partial y}&=&\frac{\partial f_2}{\partial x} \\
  \frac{\partial f_1}{\partial z}&=&\frac{\partial f_3}{\partial x} \\
  \frac{\partial f_2}{\partial z}&=&\frac{\partial f_3}{\partial y} \\
\end{eqnarray*}


If $\mathbf{F}(x,y)=\nabla\phi(x,y)$ in a 2d domain $D$, then $\mathbf{F}$ is a conservative vector field in $D$ and function $\phi$ is the potential function.
$${\frac{\partial f_{1}}{\partial y}}={\frac{\partial^{2}\phi}{\partial y\partial x}}={\frac{\partial^{2}\phi}{\partial x\partial y}}={\frac{\partial f_{2}}{\partial x}}$$
\subsubsection{Vector field in Polar Coordinates}
$$\mathbf{F}=f(r,\theta)=f_{r}(r,\theta)\mathbf{\hat{r}}+f_{\theta}(r,\theta)\mathbf{\hat{\theta}}$$
where:
$$\mathbf{\hat{r}}=\cos(\theta)i+\sin(\theta)j$$
$$\mathbf{\hat{\theta}}=-\sin(\theta)i+\cos(\theta)j$$
\subsection{Line Integral}
$$f(x,y)ds=\text{Area (tiny point)}$$
$$\text{Length of }\mathcal{C}=\int _{\mathcal{C}}f(x,y,z) \, ds=\int_{a}^{b}f(r(t)) \left|\frac{ dr }{ dt } \right| \, dt  $$
\subsubsection{Line integral of a vector field}
$$W=\int _{\mathcal{C}}F.\hat{T} \, ds =\int F \,dr =\int _{\mathcal{C}}f_{1}(x,y,z)dx+f_{2}(x,y,z)dy+f_{3}(x,y,z)dz $$
\subsection{Examples}
\subsubsection{Example 1: Conservative vector field and potential}
Determine whether the given vector field is conservative, and find a potential
function if it is:
$$\mathbf{F}(x,y,z)=( 2x y-z^{2})\mathbf{i}+( 2y z+ x^{2})\mathbf{j}-(2z x- y^{2})\mathbf{k}$$ 
\rule{\textwidth}{0.5pt}

The field is convervative if:
\begin{eqnarray*}
  \frac{\partial f_1}{\partial y}&=&\frac{\partial f_2}{\partial x} \\
  \frac{\partial f_1}{\partial z}&=&\frac{\partial f_3}{\partial x} \\
  \frac{\partial f_2}{\partial z}&=&\frac{\partial f_3}{\partial y} \\
\end{eqnarray*}
$$\frac{\partial (2xy-z^2)}{\partial y}=2x=\frac{\partial (2yz+x^2)}{\partial x} $$
$$ \frac{\partial (2xy-z^2)}{\partial z}=-2z=\frac{\partial (-2zx+y^2)}{\partial x} $$
$$\frac{\partial (2yz+x^2)}{\partial z}=2y=\frac{\partial (-2zx+y^2)}{\partial y} $$
All equations are satisfied! The field is convervative.

Find the potential function $\phi(x,y,z)$:
$$f_1=\frac{\partial \phi}{\partial x}\quad\Rightarrow\quad \phi=\int f_1dx=x^2y-z^2x+c(y,z)$$
$$\frac{\partial \phi}{\partial y}=x^2+\frac{\partial c(y,z)}{\partial y}$$
$$f_2=\frac{\partial \phi}{\partial y}=x^2+\frac{\partial c(y,z)}{\partial y}=2yz+x^2\quad\Rightarrow\quad\frac{\partial c(y,z)}{\partial y}=2yz$$ This means $c$ is a function of $y$ and $z$ and can be found by taking the anit-derivative
$$c(y,z)=y^2z+c(z)$$
Insert $c(y,z)$:
$$\phi=x^2y-z^2x+y^2z+c(z)$$
$$f_3=\frac{\partial \phi}{\partial z}=-2zx+y^2+\frac{\partial c(z)}{\partial z}=-2zx+y^2\quad\Rightarrow\quad \frac{\partial c(z)}{\partial z}=0$$
A scalar potential function of $F$:
$$\phi(x,y,z)=x^2y-z^2x+y^2z$$

\subsubsection{Example 2: Line integral}
Evaluate $\oint x^2y^2 \ dx + x^3y\ dy$ counterclockwise around the square with vertices
$(0,0)$, $(1,0)$, $(1, 1)$, and $(0, 1)$

\rule{\textwidth}{0.5pt}

Find the parameterization for each of the lines:\\
$c_1$: (0,0) to (1,0)
$$x(t)=t\qquad y(t)=0\qquad 0\leq t\leq 1$$
$$\frac{dx}{dt}=1\qqr dx=dt$$
$$\frac{dy}{dt}=0\qqr dy=0$$
$$\int_0^1(t^20^2 dt + t^3\cdot0 \cdot0)=\boxed{0}$$

$c_2$: (1,0) to (1,1)
$$x(t)=1\qquad y(t)=t\qquad 0\leq t\leq 1$$
$$\frac{dx}{dt}=0\qqr dx=0$$
$$\frac{dy}{dt}=1\qqr dy=dt$$
$$\int_0^1(1^2t^2 \cdot 0 + 1^3\cdot t dt)=\left[\frac{t^2}{2}\right]_0^1=\boxed{\frac{1}{2}}$$

$c_3$: (1,1) to (0,1)
$$x(t)=1-t\qquad y(t)=1\qquad 0\leq t\leq 1$$
$$\frac{dx}{dt}=-1\qqr dx=-dt$$
$$\frac{dy}{dt}=0\qqr dy=0$$
$$\int_0^1((1-t)^21^2 (-dt) + (1-t)^3\cdot 1 \cdot 0)=-\int_0^1(1-t)^3dt=\left[\frac{(1-t)^3}{3}\right]_0^1=\boxed{-\frac{1}{3}}$$

$c_4$: (0,1) to (0,0)
$$x(t)=0\qquad y(t)=1-t\qquad 0\leq t\leq 1$$
$$\frac{dx}{dt}=0\qqr dx=0$$
$$\frac{dy}{dt}=-1\qqr dy=-dt$$
$$\int_0^1(0^2(1-t)^2 0 + 0^3\cdot (1-t) (-dt))=\boxed{0}$$

Therefore
$$\oint x^2y^2 \ dx + x^3y\ dy=0+\frac{1}{2}-\frac{1}{3}+0=\frac{1}{6}$$
\subsubsection{Example 3: Line integral}
Evalute the line integral for $f(x,y)=x^2y^2$ along a straight line from origin to the point $(2,1)$

\rule{\textwidth}{0.5pt}

The parmeterization of arc length over t:
$$x=f(t)=t\qquad \frac{df(t)}{dt}=1$$
$$y=g(t)=2t\qquad \frac{dg(t)}{dt}=2$$
Setup integral with bounds: $0\leq t\leq 1$
$$\int_0^1 t^2(2t)^2\sqrt{f'(t)^2+g'(t)^2}dt=\int_0^1 5t^2\sqrt{5}dt$$
$$=\left[\frac{5t^3\sqrt{5}}{3}\right]^1_0=\frac{5\sqrt{5}}{3}$$

\subsubsection{Example 4: Line integral vector field}
Evaluate the line integral of the tangential compnent of the given vector field along the given curve:
$$F(x,y)=xy\mathbf{i}-x^2\mathbf{j}$$
\rule{\textwidth}{0.5pt}

For a vector field:
$$W=\int F \,dr$$
Along the line $y=x^2$:
Parametrize $x$ and $y$:
$$x(t)=t\qquad y(t)=t^2\qquad r(t)=t\mathbf{i}+t^2\mathbf{j}$$
$$\frac{dr}{dt}=\mathbf{i}+2t\mathbf{j}\quad\Rightarrow\quad dr=(\mathbf{i}+2t\mathbf{j})dt$$
Setup integral with bounds: $0\leq t\leq 1$
$$\int_{0}^{1}(t^3\mathbf{i}-t^2\mathbf{j})(\mathbf{i}+2t\mathbf{j})  dt=\int_0^1t^3-2t^3dt=\int_0^1-t^3dt$$
$$=\left[-\frac{t^4}{4}\right]_0^1=-\frac{1}{4}$$

\subsubsection{Example 5: Line integral over specified curve}
Evaluate the given line integral over the specified curve $\mathcal{C}$
$$\textstyle\int_{\mathcal{C}}(x+y)ds\qquad \mathbf{r}=a t\mathbf{i}+b t\mathbf{j}+c t\mathbf{k}\qquad\ 0\leq t\leq m$$

\rule{\textwidth}{0.5pt}

$$\int_{\mathcal{C}}f(x,y,z)ds=\int_a^bf(r(t))\left|\frac{dr}{dt}\right|dt=\sqrt{a^2+b^2+c^2}dt$$
$$ds=\left|\frac{dr}{dt}\right|dt=|a\mathbf{i}+b\mathbf{j}+c\mathbf{k}|dt$$
$$f(r(t))=at+bt$$
Solve the integral:
$$\int_0^m(at+bt)\sqrt{a^2+b^2+c^2}dt=\sqrt{a^2+b^2+c^2}\int_0^m(a+b)tdt$$
$$=\sqrt{a^2+b^2+c^2}\left[\frac{(a+b)t^2}{2}\right]_0^m=\frac{\sqrt{a^2+b^2+c^2}(a+b)m^2}{2}$$
\subsubsection{Example 6: Parametrize a curve}
Use $t=y$ to parametrize the part of the line of intersection of the two planes:\\
Plane 1: $y=2x-4$\\
Plane 2: $z=3x+1$ from $(2,0,7)$ to $(3,2,10)$

\rule{\textwidth}{0.5pt}

Find parameterization for $x$ with $y=t$:
$$t=2x-4 \qqr x(t)=\frac{t+4}{2}$$
Find parameterization for $z$ using $x(t)$:
$$z(t)=3\left(\frac{t+4}{2}\right)+1\qqr \left(\frac{3t+12}{2}\right)+1\qqr \frac{3t}{2}+7$$
The parameterization is given by:
$$r(t)=\left(\frac{1}{2}(t+4)\right)\mathbf{i}+(t)\mathbf{j}+\left(\frac{3}{2}t+7\right)\mathbf{k}$$
