\section{FIR-Filters}
\textbf{Finite Impulse Response} filters where the impulse response of the system is of finite duration.\\
A FIR filter with $N$ samples has a impulse response:
$$h(n)=\begin{cases}
a_n & \text{for } 0\leq n\leq N-1\\
0 & \text{otherwise}
\end{cases}$$
Therefore the filter has $N$ coefficients $a_n$ for $n=0,1,\cdots,N-1$.
\subsection{Linear phase}
\subsection{Frequency response}
\subsection{FIR-Filter design} 
If there is $N$ samples in the impulse response, then:
$$M=\frac{N-1}{2}$$
And the transfer function of the filter is:
$$H(z)=\frac{Y(z)}{X(z)}=\sum_{i=0}^{2M}a_iz^{-i}$$
\begin{table}[h]
\centering
\begin{tabular}{|>{\columncolor[HTML]{C0C0C0}}l |c|c|c|}
\hline
\textbf{Filter type}& $\mathbf{c_0}$ & $\mathbf{c_m=c_{-m}}$ & $a_i$ \\ \hline
Low-pass& $2Tf_a$ &${\frac{1}{m\pi}}\sin(2\pi m T f_{a})$  & $c_{M-i}$ \\ \hline
 High-pass& $1-2Tf_a$ & $\frac{1}{m\pi}(\sin(m\pi)-\sin(2\pi m T f_{a}))$ & $c_{M-i}$ \\ \hline
 Band-pass& $2T(f_{a_2}-f_{a_1})$ &$\frac{1}{m\pi}{\bigl(}\sin(2\pi m T f_{a_{2}})-\sin(2\pi m T f_{a_{1}}){\bigr)}$  & $c_{M-i}$ \\ \hline
 Band-stop&$1-2T(f_{a_2}-f_{a_1})$  & $\frac{1}{m\pi}(\sin(m\pi)+\sin(2\pi m T f_{a_{1}})-\sin(2\pi m T f_{a_{2}}))$ &  $c_{M-i}$\\ \hline
\end{tabular}
\end{table}

Using the inverse z-transform, the impulse response is found:
$$y(n)=\sum_{i=0}^{2M}a_i x(n-i)$$


\subsection{Window functions}
Due to the ripples in the passband and stopband, the ideal filter is not realizable. Therefore, the ideal impulse response is multiplied by a window function $w(n)$ to obtain a realizable impulse response $h(n)$:
$$h(n)=h_\infty(n) w(n)$$

Following are some common window functions:
\begin{table}[H]
\centering
\begin{tabular}{|>{\columncolor[HTML]{C0C0C0}}l |c|c|c|c|}
\hline
\textbf{Window}& $\mathbf{B_n}$ & $\mathbf{M_{min}}$ & Min. stopband attenuation & Min. passband ripple \\ \hline
Rectangular         & 2 &$f_s/\Delta_f$  & 20 dB &1.5 dB\\ \hline
Barlett             & 4 & $2f_s/\Delta_f$ & 25 dB &0.1 dB\\ \hline
Hamming             & 4 &$2f_s/\Delta_f$  & 50 dB &0.05 dB\\ \hline
Hanning             &4  & $2f_s/\Delta_f$ & 45 dB&0.1 dB\\ \hline
Kasier ($\beta=\pi$) &2.8  & $1.4f_s/\Delta_f$ & 40 dB&0.2 dB\\ \hline
Kasier ($\beta=2\pi$)&4.4 & $2.2f_s/\Delta_f$ &  65 dB&0.01 dB\\ \hline
\end{tabular}
\end{table}

The new fouirer coefficients $c'_m$ if $-M\leq m \leq M$:
$$c'_m=c_mw_m$$
Therefore 
$$a_i=c'_{M-i}$$

\subsubsection{Rectangular window}
$$w(n)=\begin{cases}
1 & \text{if } -M\leq n\leq M\\
0 & \text{otherwise}
\end{cases}$$
\subsubsection{Barlett window}
$$w(n)=\begin{cases}
  1-\frac{|n|}{M} & \text{if } -M\leq n\leq M\\
  0 & \text{otherwise}
\end{cases}$$
\subsubsection{Hamming and Hanning window}
$$w(n)=\begin{cases}
\alpha+(1-\alpha)\cos\left(\frac{n\pi}{M}\right) & \text{if } -M\leq n\leq M\\
0 & \text{otherwise}
\end{cases}$$
where $\alpha=0.54$ for Hamming and $\alpha=0.5$ for Hanning.
\subsubsection{Kaizer window}
$$w(n)=\begin{cases}
  \frac{I_0\left(\beta\sqrt{1-\left(\frac{n}{M}\right)^{2}}\right)}{I_0(\beta)}& \text{if } -M\leq n\leq M\\
0 & \text{otherwise}
\end{cases}$$

\subsection{Examples}
\subsubsection{Example 1: High-pass filter}
A FIR high-pass filter with cutoff frequency $f_{a}=\SI{1}{\kilo\hertz}$, a trasisition band of $\Delta f\leq\SI{0.5}{\kilo\hertz}$, maximum stopband attenuation of $H_s\leq\SI{-50}{\decibel}$ and sample frequency $f_s=\SI{5}{\kilo\hertz}$ is designed.

\textbf{Which window function should be used?}\\
To satisfy the stopband attenuation, the Hamming or the Kaiser window should be used, as they both have 50 dB or higher stopband attenuation.
The Hamming window is chosen, which has a $B_n=4$ and minimum stopband attenuation of 50 dB.

\textbf{What is the minimum filter order?}\\
The filter order is given by $2M$, where $M$ is:
$$M=\frac{B_nf_s}{2\Delta f}=\frac{4\cdot5000}{2\cdot 500}=20$$
Therefore the minimum filter order is $2M=40$.

\textbf{What is the coefficients for the filter?}
$$c_0=1-2Tf_a\qquad c_m=\frac{1}{m\pi}(\sin(m\pi)-\sin(2\pi m T f_{a}))$$
And the Hamming window ($\alpha=0.54$):
$$w_m=\alpha+(1-\alpha)\cos\left(\frac{m\pi}{M}\right)$$
Calculating the coefficients for the filter without window:
\begin{eqnarray*}
c_0&=&1-2\cdot\frac{1}{5000}\cdot1000=0.6\\
c_1=c_{-1}&=&\frac{1}{1\cdot\pi}\left(\sin(1\cdot\pi)-\sin(2\pi\cdot 1\cdot\frac{1}{5000}\cdot1000)\right)=-0.3027\\
c_2=c_{-2}&=&\frac{1}{2\cdot\pi}\left(\sin(2\cdot\pi)-\sin(2\pi\cdot 2\cdot\frac{1}{5000}\cdot1000)\right)=-0.0935\\
c_3=c_{-3}&=&\frac{1}{3\cdot\pi}\left(\sin(3\cdot\pi)-\sin(2\pi\cdot 3\cdot\frac{1}{5000}\cdot1000)\right)=0.0624\\
\end{eqnarray*}
And so on... to $c_M=c_{20}$

Calculating the coefficients for the window:
$$w_0=0.54+(1-0.54)\cos\left(\frac{0\cdot\pi}{20}\right)=1$$
$$w_1=w_{-1}=0.54+(1-0.54)\cos\left(\frac{1\pi}{20}\right)=0.9943$$
$$w_2=w_{-2}=0.54+(1-0.54)\cos\left(\frac{2\pi}{20}\right)=0.9775$$
$$w_3=w_{-3}=0.54+(1-0.54)\cos\left(\frac{3\pi}{20}\right)= 0.9499$$
And so on... to $w_M=w_{20}$

Calculate the final coefficients $a_i=c_{M-i}w_{M-i}$:
$$a_{20}=c_{20-20}w_{20-20}=c_{0}w_{0}=0.6\cdot 1=0.6$$
$$a_{19}=c_1w_1=-0.3027\cdot 0.9943=-0.3010$$
$$a_{18}=c_2w_2=-0.0935\cdot 0.9775=-0.0914$$
$$a_{17}=c_3w_3=0.0624\cdot 0.9499=0.0592$$
And so on... from $a_0 \to a_{2M}=a_{40}$

\subsubsection{Example 2: Band-stop filter}
Find the filter coefficients for a FIR band-stop filter without a window. 
The filter has cutoff frequencies: $f_{a_1}=1\text{ kHz }f_{a_2}=2\text{ kHz}$ and a sample frequency of $f_s=10\text{ kHz}$.
The filter shall have 5 samples.

\rule{\textwidth}{0.5pt}

For a band-stop filter:
$$c_0=1-2T(f_{a_2}-f_{a_1})\qquad c_m=\frac{1}{m\pi}(\sin(m\pi)+\sin(2\pi m T f_{a_{1}})-\sin(2\pi m T f_{a_{2}}))$$
Calculate the coefficients:
$$T=\frac{1}{f_s}=0,0001$$
$$M=\frac{N-1}{2}=2$$
$$c_0=1-2T(f_{a_2}-f_{a_1})=1-2(0,0001)(2000-1000)=1-0.2=\boxed{0.8}$$
$$c_m=\frac{1}{m\pi}(\sin(m\pi)+\sin(2\pi m (0,0001) 1000)-\sin(2\pi m (0.0001) 2000))$$
$$=\frac{1}{m\pi}(\sin(m\pi)+\sin(2\pi m (0.1))-\sin(2\pi m (0.2)))$$
$$c_1=c_{-1}=\frac{1}{1\pi}(\sin(1\pi)+\sin(2\pi 1 (0.1))-\sin(2\pi 1 (0.2)))$$
$$=\frac{1}{\pi}(\sin(0.2\pi)-\sin(0.4\pi)=\boxed{-0.1156}$$
$$c_2=c_{-2}=\frac{1}{2\pi}(\sin(2\pi)+\sin(2\pi 2 (0.1))-\sin(2\pi 2 (0.2)))$$
$$=\frac{1}{2\pi}(\sin(0.4\pi)-\sin(0.8\pi)=\boxed{0.0578}$$
The coefficients are ($a_i=c_{M-i}$):
$$a_0=0.0578$$
$$a_1=-0.1156$$
$$a_2=0.8$$
$$a_3=-0.1156$$
$$a_4=0.0578$$
