\section{Komparatorer og multivibratorer}

An ideal comparator compares two input voltages and produces a logic output signal whose value (high or low) depends on which
of the two inputs is largest. For a comparator it has an inverting and a non-inverting input, and the output is a digital signal.
If the voltage $v_1$ applied on t the noninverting input is larger than the voltage $v_2$ applied to the inverting input, the output
will be high, and vice versa.

Comparators are designed to minimize the time delay between the input signal and the output signal.

\textbf{Schmitt trigger} is a comparator with hysteresis, which means that it has two different threshold levels for rising and falling edge.
A schmitt trigger uses positive feedback to prevent oscillation and to increase the speed of the switching.

\textbf{Astable multivibrator} is a switching oscillator, that can be formed by adding an RC feedback network to a Schmitt trigger.
These circuits are useful for generating relatively low frequency square waves. The output of the multivibrator is a square wave.

When designing an astable multivibrator you should think of  asmall capacitance calls for a large resistance. However, using a large resistance
leads the charging current to be small and can be significantly affected by the bias current of the comparator, which is unpredictable.

\textbf{555 Timer}

The 555 timer IC is economical and convenient for use in multivibrator circuits because few external components are required.



\textbf{A 555 monostable multivibrator} is a circuit that produces an output pulse of fixed duration in response to a trigger signal.
Monostables are useful in producing timing signals. For example, automatic garage-door openers often have a light that is turned on when the door is opened.
The light stays on for a few minutes and then automatically turns off.

For a 555 monostable multivibrator circuit only  two external components are needed: a resistor and a capacitor.
