\section{Implementation}

\subsection{Anti-windup}

Real electromechanical systems has saturations on their physical variables. In particular,
the output of any actuator is limited from above and below. (Clipping).
If you stay between the limits, the system will behave as a linear system.

\textbf{Back-calculation}

An anti-windup scheme that recomputes the integral term when the system
input is saturated. The value of the integrator output is not changed instantaneously, but it is changed
based on the tracking time constant.

\begin{center}
	\includegraphics[width = 0.8\textwidth]{Images/backCalc.png}
\end{center}

The input to the integrator is given by
$$ \frac{1}{T_t}e_s  + \frac{K_p}{T_i}e $$

Where $e_s$ is zero when the system is not saturated. In steady state, the output of the integrator is constant:
hence, its input must be zero, i.e.

$$ e_s = -\frac{K_p T_t}{T_i}e $$


\textbf{Conditional integration}

Also known as clamping is a bit simpler, and just stops integreating when the system is in saturation.
\begin{center}
	\includegraphics[width=0.7\textwidth]{Images/conditional-integration.png}
\end{center}


\textbf{Setpoint weighting}

To avoid large control signals when changing the reference rapidly, setpoint weiths can be introduced.
These modify the PID controller to be:

$$ u(t) = K_p(\beta r(t) -y(t)) + K_i \int_0^t (r(\tau)-y(\tau))d\tau + K_d (\gamma \frac{dr}{dt} - \frac{dy}{dt})$$

where $\gamma \text{and} \beta$ are setpoint weights. Typically $\gamma = 0$ and $\beta$ takes values between
0 and 1.

\textbf{Filtering}

To avoid large control signal noise on the measured output $y$, one adds a filter
on the derivative term. Often a first order filter is sufficient, but this is application dependent.

$$ u_d = k_d s \frac{1}{1+sT_f}$$

Sometimes it is favorable to filter the control signal directly, thus the controller gets the form
$$ C(s) = K_p (1+\frac{1}{sT_i}+sT_d) \frac{1}{1+sT_f+(sT_f)^2/2}$$

\subsection{Digitalization}
Most control systems are sample data systems, i.e., they consist of both discrete and continous signals.
The sampling frequency used for the discrete controlelr should be above 20 times the closed-loop bandwidth.


\subsubsection{Emulation}
Design continous controller $K(s)$ and approximate it with $K(z)$ obtained via e.g. Tustin's method.
Tustin's method also called the trapezoidal rule means to replace the variable $s$ with $$\frac{2}{T} \frac{z-1}{z+1}$$


The control output of the discrete PID controller can be written using 3 terms.

$$u_p(kT+T) = k_p e(kT+T)$$
$$u_I(kT+T) = u_I(KT)+K_i \frac{T}{2} (e(kT+T)+e(kT))$$
$$u_D(kT+T) = k_D \frac{2}{T} (e(kT+T)-e(kT))-u_D(kT)$$

$$u(kT+T) = u_p(kT+T)+u_I(kT+T)+u_D(kT+T)$$

To analyze the system, the I-term and D-term are z-transformed:
$$z u_I(z) = u_I(z) + k_i \frac{T}{2} (ze(z)+e(z))$$
$$z u_D(z) = k_D \frac{2}{T} (ze(z)-e(z))-u_D(z)$$

This gives the following expression for the controller
$$u(z) = (k_P+k_I \frac{T}{2} \frac{z+1}{z-1}+k_D \frac{2}{T} \frac{z-1}{z+1})e(z)$$


\textbf{Compensation for sampling effects}

\newpage
\textbf{Design procedure}

A discrete controller can be designed by emulation for the system $G(s)$ according to the next procedure:

\begin{enumerate}
	\item{Design continuous compensation for the system $G_d(s)G(s)$, where $G_d(s)$ approximates a delay of $T/2$}
	\item{Derive the discrete controller by applying Tustin's rule or the matched pole-zero method}
	\item{Analyze the design by simulation or experimentally}
\end{enumerate}



\subsubsection{Numerical Integration Methods}





\subsubsection{Discrete Design}
Design the discrete controller directly, without computing $K(s)$ first.

\subsection{Examples}

\textbf{Emulation example of PID controller}

$$K(s) = 1.4 \frac{s+6}{s}$$
Use Tustin's method to emulate the controller in discrete time.
$$K(z) = K(\frac{2}{T} \frac{z-1}{z+1}) = 1.4 \frac{\frac{2}{T} \frac{z-1}{z+1}+6}{\frac{2}{T} \frac{z-1}{z+1}}
	= 1.4 \frac{(1+3T)z-(3T-1)}{z-1}$$

In practice you implement the difference equation - not a discrete transfer function.
