\section{Linear Time Invariant Systems}

\textbf{Linear Map} \newline
The map $f: R^n -> R^m$ is said to be linear if for any $x,y \epsilon R^n$ and
$\alpha \epsilon R$, the following conditions hold

$$f(x+y) = f(x) + f(y)  \qquad \text{Super position}$$
$$f(ax) = \alpha f(x) \qquad \text{Homogeneity}$$

The function has to go through (0,0) in 2D for it to be linear due to homogeneity.


\textbf{Time-Invariant System} \newline
Let \sigma:$ R x R^m -> R^p$ define the input-output behavior of a system model \SUM.
The system \SUM is time-invariant if for any input signal $u:R->R^m$
and any delay $\tau \epsilon R$ the following relation holds:
$$ y(t-\tau)= \sigma(t,u(t-\tau))$$
for all times $t\epsilon R$, where $y$ denotes the output signal of the system.

The importance is that the system does not change its behavior due to time.
This can be seen as a canon firing at 8am it will not fire different
compared to if you do the same at 5pm.
\subsection{Examples}
