\section{Observer Design}
Is used when you can't measure all states.

\subsection{Observability}
Estimation of state.

Tells when it is possible to estimate the states of a system from the output.
An observer can only be made if the system is observable.


\textbf{Observability matrix}

$$O = \begin{bmatrix} C \\ CA \\ CA^2 \\ \vdots \\ CA^{n-1} \end{bmatrix}$$

Has to have full rank to be observable.


\subsection{Full Order Observer}
hat symbolizes the estimated value.

Estimation of all states. (tilstandsestimator)

$$\dot{\hat{x}} = A\hat{x} + Bu + L(C\hat{x}-y)$$
$$\hat{y} = C\hat{x}$$

Error, $e = \hat{x} - x$:
$$\dot{e} = \dot{\hat{x}} - \dot{x} = A\hat{x}+Bu+L(C\hat{x} - y) - (Ax+Bu) $$
$$ = A(\hat{x}-x) + L(C\hat{x}-Cx)$$
$$ = (A+LC)e$$

Eigenvalues for A+LC should be placed in the left half plane for the error to converge to zero.
This has to be done by only changing L.

\textbf{Theorem}

A full order observer for the system with observer gain $L$ is stable,
if and only if the eigenvalues of the matrix $A+LC$ all have a negative real part.
Moreover, such an $L$ always exists, if (A,C) is observable.


\textbf{Observable Canonical Form}

Any observable single output system can be written in the form:



\subsection{Observer Design}

Designing an observer needs an observer gain matrix $L$. This can be found by




\subsection{Observer Based Control}

Combination of observer and state feedback.
This gives 2n poles because of the extra eigenvalues.
The state feedback poles should be placed 2 to 6 times further
to the left than the observer poles. This is done to achieve a faster response.

\textbf{Seperation principle}:

You can design the observer and the controller independently and then combine them.
An observer with observer gain $L$ and feedback gain $F$ results in 2n closed loop poles,
coinciding with the eignevalues of $A+BF$ and $A+LC$.



\subsection{Examples}
