\section{Introduction to Control}

\subsection{Open Loop Control}
(åbensløjfe regulering)

Steady-State Value of Time Function:

Suppose that Y(s) is the Laplace transform of y(t). Then the final value of y(t) is either:
\begin{itemize}
	\item{Ubounded. If Y(s) has any poles in the open right half-plane (unstable)}
	\item{Undefined. If Y(s) has a pole pair on the imaginary axis.}
	\item{Constant. If all poles of Y(s) are in the open left half-plane, except for one at s=0}
\end{itemize}

\textbf{The Final Value Theorem (slutværdi-sætningen)}

If all poles of sY(s) are in the open left half-plane, then:
$$\lim_{t \to \infty} y(t) = \lim_{s \to 0} sY(s)$$

The Final Value Theorem determines the constant value that the impulse response of a stable system converges to.
The theorem can also be used to determine the DC gain of a system, i.e., the output when a step input is applied to the system.

By integrating an impulse response, the step response is obtained. The step response is the integral of the impulse response,
and the impulse response is the derivative of the step response.



\subsection{Feedback Control}
(tilbagekobling)
\subsection{Ready State Tracking}



\subsection{Cascase Control}







\subsection{Examples}
