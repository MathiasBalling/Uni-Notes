\section{Kalman Filter}

\subsection{Random Values}

Probability density function is used for determining
the probability of a random variable falling within a particular range of values.
This is done because a continuous random variable X often has zero probability of being one particular value.

$$ Pr(a \leq X \leq b) = \int_{a}^{b} f_X(x) dx $$


The expected value (mean value) of a continuous random variable X
can be determined from the probiability density function $f_X(x)$ as follows:

$$ E[X] = \int_{-\infty}^{\infty} x f_X(x) dx $$

The variance of X is defined as:

$$ Var[X] = E[(X - E[X])^2] $$


For a multivariate random variable X, the expected value is a vector:
$$ X = \begin{bmatrix} X_1 \\ X_2 \\ \vdots \\ X_n \end{bmatrix} $$

Covariance matrix

$$ \sum = E[(X - E[X])(X - E[X])^T] $$

The variance for the different values can be found on the diagonal of the covariance matrix.



\subsection{Kalman Filter}



\subsection{Examples}

