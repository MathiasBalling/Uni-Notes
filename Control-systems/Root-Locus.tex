\section{Root Locus}
Rudkurve metoden.
Grafisk metode til desgin af regulatorer.

\textbf{Rule 1}

Consider the characteristic equation:
$$ 1+KG(s) = 1 + K\frac{Q(s)}{P(s)} = 0$$
This can be written as:
$$P(s) + KQ(s) = 0$$

This is a polynoial of degree $N = max(m,n)$ where m is the number of poles and n is the number of zeros.
Lemma. A univariate polynomial of degree d has d roots in $\mathbb{C}$ .

There are $N$ lines (loci) where $N=max(m,n)$. Where m is the number of poles and n is the number of zeros.

\textbf{Rule 2}

Similar to previously, the characteristic equation is rewritten as:
$$P(s) + KQ(s) = 0$$
Let $K=0$, then we observe that the roots of the characteristic equation are the poles of
the open-loop system.

Let $K\to \infty$, then we observe that the roots of the characteristic equation are the zeros of
the open-loop system.

$$\frac{P(s)}{K} + Q(s) = 0$$

Defintion: As $K$ increases from 0 to $\infty$, the closed loop roots move from the open loop poles of $G(s)$
to the open-loop zeros of $G(s)$.

\textbf{Rule 3}

When roots are complex they occur in conjugate pairs.

\textbf{Rule 4}

We study the rewritten characteristic equation:
$$\frac{Q(s)}{P(s)} = -\frac{1}{K}$$
And see that the phase of $\frac{Q(s)}{P(s)}$ is $180^\circ$ to satisfy the equation.

The transfer function can be written as:
$$T(s) = \frac{Q(s)}{P(s)} = \frac{(s-z_1)(s-z_2)\cdots(s-z_m)}{(s-p_1)(s-p_2)\cdots(s-p_n)}$$
where $z_i$ are the zeros and $p_i$ are the poles.
Let $z \in \mathbb{C}$ then $zz* = |z|^2$; hence, complex pole pairs and pairs of complex
conjugated zeros do not affect the phase of $T(s)$ for $s \in \mathbb{R}$.

The phase of $(s-z_m)$ when $s,z_m \in \mathbb{R}$ is
\begin{equation}
	\angle(s-z_m) =
	\begin{cases}
		180^\circ & \text{if } s > z_m \\
		0^\circ   & \text{otherwise}
	\end{cases}
\end{equation}
The portion of the real axis to the left of an odd number of open loop poles and zeros
are part of the loci.

\textbf{Rule 5}

Lines leave and enter the real axis at $90^\circ$ angles.

\textbf{Rule 6}

For very large values of $s$ the equation:
$$1+K\frac{s^m+b_1s^{m-1}+\cdots+b_m}{s^n+a_1s^{n-1}+\cdots+a_n} = 0$$
can be approximated to:
$$1+K\frac{1}{(s-\alpha)^{n-m}}$$

The phase of this expression should be $180^\circ$ for this expression to hold.
This implies that:
$$(n-m)\phi_l = 180^\circ+360^\circ(l-1)$$
where $\phi_l$ is the phase of the expression.

$$\phi_l = \frac{180^\circ+360^\circ(l-1)}{n-m}$$


let $m<n-1$ then:
$$-\sum r_i = -\sum p_i$$
where $r_i$ are the closed-loop poles and $p_i$ are the open-loop poles.

For $s$ going to $\infty$, it is known that $m$ closed-loop poles go towards the open-loop zeros,
and $n-m$ closed-loop poles go towards $\alpha$, i.e.
$$-\sum r_i = -(n-m)\alpha - \sum z_i = -\sum p_i$$
where $z_i$ are the open-loop zeros.

Thus,
$$\alpha = \frac{\sum z_i - \sum p_i}{n-m}$$

\subsection{Examples}


\textbf{Departure angle}

$q=$ multiplicity of the pole. This means that the pole is repeated $q$ times
at the same location.
