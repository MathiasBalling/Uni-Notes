\section{The Nyquist Stability Criterion}

\subsection{Frequency response}
The frequency response of a system is the steady-state response of a system to a sinusoidal input.
The output of a time-invariant system will have the same frequency as the sinusoidal input,
but possibly with a different amplitude and phase.

The output of a time-invariant system is given by:
$$y(t) = \int_{-\infty}^{\infty} h(\tau)u(t-\tau)d\tau$$
Where y is the output, u  is the input, and h is the impulse response of the system.

To obtain the frequency response, only sinusoidal inputs are considered.

The frequency response of H(s) is given by the magnitude and phase of $H(j\omega)$

$$M = |H(j\omega)|$$
$$\phi = \angle H(j\omega)$$
The M is the amplitude ratio and $\phi$ is the phase shift.

The bandwidth of a closed-loop system $T(s)$ is defined to be the maximum frequency at which
the output y of a system will track a sinusoidal input $r$ in a satisfactory manner. Output attaenuated to $1/\sqrt{2}$ of the input amplitude.
Formally the bandwidth $\omega_{BW}$ of $T(s)$ is the maximal frequency such that:

$$T(j\omega) >= 1/\sqrt{2}$$

The maximal value of the frequency response is called the resonant peak  $M_r$


\subsection{Examples}
