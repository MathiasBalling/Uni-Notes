\section{PDE}
Partial Differential Equations are equations with multiple variables and derivatives.
They are used to model many physical phenomena, such as heat, sound, and light.
The totality of solutions to a PDE is called its general solution, and there can be a lot.

\subsection{Classification of PDEs}
General representation of a PDE:
\[
  A\frac{\partial^2u}{\partial x^2}+B\frac{\partial^2u}{\partial x\partial y}+C\frac{\partial^2u}{\partial y^2}+D\frac{\partial u}{\partial x}+E\frac{\partial u}{\partial y}+Fu=G
\]
\[
  Au_{xx}+Bu_{xy}+Cu_{yy}+Du_x+Eu_y+Fu=G
\]
Conditions:
%todo: Is this correct?

$$\begin{array}{l}
\textbf{Linear: }A,B,C,D,E,F \text{ are only function of x,y variables, not u.}\\
\textbf{Quasi-linear: }A,B,C,D,E,F \text{ may be function of } (x,y,u,u_x,u_y)\\
\textbf{Fully non-linear: }A,B,C,D,E,F \text{ may be function of } (x,y,u,u_x,u_y,u_{xx},u_{yy},u_{xy})
\end{array}$$

\subsection{Characteristics of PDEs}
$$\begin{array}{llll}
  B^2-4AC>0&2\text{ real roots}&2\text{ characteristics}&\textbf{Hyperbolic PDE}\\
  B^2-4AC=0&1\text{ real roots}&1\text{ characteristics}&\textbf{Parabolic PDE}\\
  B^2-4AC<0&0\text{ real roots}&0\text{ characteristics}&\textbf{Elliptic PDE}\\
\end{array}$$
Tyoes of varius PDEs:

\textbf{Wave Equation:} Hyperbolic PDE

\textbf{Heat Equation:} Parabolic PDE

\textbf{Laplace Equation:} Elliptic PDE

\subsection{Important Second-Order PDEs}

$$\begin{array}{ll}
  \displaystyle \frac{\partial^{2}u}{\partial t^{2}}=c^{2}\,\frac{\partial^{2}u}{\partial x^{2}}&\text{One-dimensional wave equation}\\ \\
  \displaystyle{\frac{\partial u}{\partial t}}=c^{2}{\frac{\partial^{2}u}{\partial x^{2}}}&\text{One-dimensional heat equation}\\ \\
 \displaystyle {\frac{\partial^{2}u}{\partial x^{2}}}+{\frac{\partial^{2}u}{\partial y^{2}}}=0&\text{Two-dimensional Laplace equation}\\ \\
 \displaystyle {\frac{\partial^{2}u}{\partial x^{2}}}+{\frac{\partial^{2}u}{\partial y^{2}}}=f(x,y)&\text{Two-dimensional Poisson equation}\\ \\
 \displaystyle \frac{\partial^{2}u}{\partial t^{2}}=c^{2}\biggl(\frac{\partial^{2}u}{\partial x^{2}}+\frac{\partial^{2}u}{\partial y^{2}}\biggr)&\text{Two-dimensional wave equation}\\ \\
 \displaystyle {\frac{\partial^{2}u}{\partial x^{2}}}+{\frac{\partial^{2}u}{\partial y^{2}}}+{\frac{\partial^{2}u}{\partial z^{2}}}=0&\text{Three-dimensional Laplace equation}
\end{array}$$
\subsection{Initial and Boundary Conditions}

\subsection{Wave Equation (1D)}
One dimensional wave equation is given by:
\begin{equation}
  \frac{\partial^2 u}{\partial t^2} = c^2 \frac{\partial^2 u}{\partial x^2}\quad\quad\text{where }c^2=\left[\frac{T(x,t)}{\mu_x}\right]
  \label{eq:wave1d}
\end{equation}
With two boundary conditions $x=0$ and $x=L$:
$$u(0,t)=0\qquad\quad u(L,t)=0\qquad\qquad \text{For all }t\geq 0$$
And two initial conditions, initial displacement and initial velocity at time $t=0$:
$$u(x,0)=f(x)\qquad\quad \frac{\partial u}{\partial t}(x,0)=g(x)\qquad\qquad \text{For all }0\leq x\leq L$$
Steps to solve:
\begin{enumerate}
  \item Method of Separation of Variables $\quad u(x,t)=X(x)T(t)$
  \item Satify the Boundary Conditions test
  \item Fourier Series Validation

\end{enumerate}
\subsubsection{D’Alembert’s Solution of the Wave Equation}
His solution is given by \cref{eq:wave1d} but extended to two variables:
\begin{equation}
  v=(x-ct) \quad\quad\quad\quad w=(x+ct)
  \label{eq:dalambert}
\end{equation}
I.e. $u(v,w)$. Partial derivatives from chain rule:
\[
  u_x=u_v \cdot v_x+u_w\cdot w_x=u_v+u_w
\]
For double derivatives:
\[
  u_{xx}=(u_v+u_w)_x=(u_v+u_w)_v v_x+(u_v+u_w)_w w_x=u_{vv}+2u_{vw}+u_{ww}
\]
With respect to $t$:
\[
  u_{tt}=c^2u_{xx}=c^2(u_{vv}+2u_{vw}+u_{ww})
\]
From \cref{eq:wave1d} and \cref{eq:dalambert}:
\[
  u_{vw}=\frac{\partial^2u}{\partial w \partial v}=0
\]
This can be solved by integrating with respect to $v$ and $w$:
\[
  \frac{\partial u}{\partial v}=h(v) \text{ and } u=\int h(v)\ dv + \psi(w)
\]
Here, $h(v)$ and $\psi(w)$ are arbitrary functions of $v$ and $w$, respectively.
The solution in term for $x$:
\[
  u=\phi(v)+\psi(w)
\]
This is d'Alembert's solution, which is the general solution to the wave equation.
\[
  u(x,t)=\phi(x+ct)+\psi(x-ct)
\]
This solution satisfies the wave equation and the initial conditions:




\subsection{Heat Equation (1D)}
\begin{equation}
  \frac{\partial u}{\partial t}=c^2\frac{\partial^2 u}{\partial x^2}
  \label{eq:Heat1D}
\end{equation}

Conditions:
\begin{itemize}
  \item PDE is linear and homogeneous.
  \item Boundary conditions are linear and homogeneous.The two for $u(x,t)$ is $u(0,t)=0$ and $u(L,t)=0$ for all $t>0$.
  \item One initial condition at time ($t=0$): $u(x,0)=f(x)$.
\end{itemize}
Solve the 

Solution:
$$u(x,t)=\sum_{n=1}^{\infty}B_n\sin\left(\frac{n\pi x}{L}\right)e^{-\lambda_n^2t}$$
where
$$\lambda_n = \frac{cn\pi}{L}$$
and
$$B_n = \frac{2}{L}\int_0^Lf(x)\sin\left(\frac{n\pi x}{L}\right)dx\quad\quad \text{for }n=1,2,3\dots$$
\subsection{Examples}
\subsubsection{Example 1: Type, Normal Form, and solve}
Find the type, transform to normal form, and solve.
$$u_{xy}-u_{yy}=0$$

\rule{\textwidth}{0.5pt}

Find $A,B,C$:
$$Au_{xx}+2Bu_{xy}+Cu{yy}=f(x,y,u,u_x,u_y)$$
$$A=0\qquad 2B=1\Rightarrow B=\frac{1}{2}\qquad C=-1$$

Find the type:
$$B^2-4AC=\left(\frac{1}{2}\right)^2-4\cdot 0\cdot -1=\frac{1}{4}$$
Since $B^2-4AC>0$ the PDE is hyperbolic.

Transform to normal form:
$$Ay''-2By'+C=0\quad\Rightarrow\quad-y'-1=0\quad\Rightarrow\quad y'=-1$$
$$y=\int \frac{dy}{dx}=\int -1dx=-x+c_1$$
$$c_1=x+y$$

Transform the variables:\\
Only one constant, therefore $v=x$
$$v=x\qquad v_x=1\qquad v_y=0$$
$$w=x+y\qquad w_x=1\qquad w_y=1$$
$$u_x=u_v v_x+u_w w_x=u_v \cdot 1+u_w \cdot 1=u_v+u_w$$
$$u_{xy}=(u_v+u_w)_v v_y+(u_v+u_w)_w w_y=u_{xy}=(u_v+u_w)_v \cdot 0+(u_v+u_w)_w \cdot 1=\boxed{u_{vw}+u_{ww}}$$
$$u_y=u_v u_y + u_w w_y=u_v \cdot 0 + u_w \cdot 1=u_w$$
$$u_{yy}=(u_w)_v v_y+(u_w)_w w_y=(u_w)_v \cdot 0+(u_w)_w \cdot 1=\boxed{u_{ww}}$$
The normal form is:
$$u_{vw}+u_{ww}-u_{ww}=0\quad\Rightarrow\quad \boxed{u_{vw}=0}$$

Solve:
$$u_{vw}=0\quad\Rightarrow\quad u_v=h(v)$$
$$u=g(w)+\int h(v)$$
$$u(v,w)=g(w)+f(v)$$
Insert $x,y$:
$$u(x,y)=g(x+y)+f(x)$$
Where $f$ and $g$ er orbitrary functions.

\subsubsection{Example 2: Type, Normal Form, and solve}
Find the type, transform to normal form, and solve.
$$u_{xy}-u_{yy}=0$$

\rule{\textwidth}{0.5pt}
