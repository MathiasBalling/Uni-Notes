\section{ARM architecture}

Acorn RISC Machine. Reduced Instruction Set Computer, later Advanced RISC Machine.
Key feature is that ARM processors have good performance per Watt.



\subsection{RISC}

RISC is a small set of simple and general instructions.
Advantages compared to CISC:
\begin{itemize}
	\item Instructions take one clock cycle
	\item Performance is better due to simplified instruction sets
	\item Less chip space is used due to reduced instruction set
	\item Can easily be designed as compared to CISC
	\item Reduced per chip cost, as it uses smaller chips
\end{itemize}

\begin{itemize}
	\item Performance of the processor will vary according to the code being executed
	\item RISC processors require very fast memory systems to feed various instructions
\end{itemize}

Risc is used in Qualcomm snapdragon which is in most smartphones.
In addition to this apple silicon is also based on ARM.


\subsection{ARM Cortex-M4 peripherals}

SysTick timer is a 24-bit timer that counts down to zero.


\textbf{Registers:} A processor register is one of a small set of data holding places
that are part of the computer processor. A register may hold an instruction,
a storage address, or any kind of data (such as a bit sequence or individual chracters).
Some  instructions specify registers as part of the instruction.

\textbf{Data types:} 32-bit words, 16-bit half words, 8-bit bytes.

\textbf{Interrupt vector table:} Contains the addresses of the interrupt service routines (ISR).
Using the addresses you can jump to the ISR when an interrupt is triggered.
