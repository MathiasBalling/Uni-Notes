\section{RTCS}


Non-preemptive scheduling method.
Each task runs until it has finished its job.
Scheduling is based on the tick. The tick is the basic
unit of time for task scheduling.

RTCS features tasks, task states, task events, timers, semaphores and queues.

\textbf{API} (Application programming interface): A set of functions and procedures
allowing the creation of applications that access the features or data of
an operating system, application, or other service.

In RTCS this includes Queue API, Semaphore API, Timer API, Task API and Event API.

A superloop is placed within the schedule function.

A task may communicate by reading and writing files. A file can be
a data set that you can read and write repeatedly. A stream of bytes
generated by a program. Or a stream of bytes received from or sent to
a peripheral device.

Files have been used in the lectures by setting up a pool of files
consisting of files for uart, LCD and keyboard. Generally
I/O interaction is done through files. This does not involve
GPIO pins of the controller.

