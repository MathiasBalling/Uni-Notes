\section{Serial Communication}


\subsection{UART}

Serial port is a serial communication phystical interface
through which information transfers in or out one bit at a time.
Data transfer through serial ports connected the computer
to devices such as terminals and variuos peripherals.

\textbf{Baudrate}

It is necessary to choose a proper oscillator to get the
correct baud rate with little or no error. Therefore, some common buad
rates are known. E.g. 9600.

LSB is sent first known as little-endian. Also possible but
rarely used is big-endian or MSB first.


When setting up UART the clock frequency of the FPGA should be 16 times faster
than the baud rate.

UART uses parity check.

\subsection{SPI}

Serial Peripheral Interface is a synchronous serial communication interface
for short-distance communication.
Full duplex, single master.



\subsection{Inter IC}

Serial synchronous bus. Widely used to connect low-speed ICs processing
units in short-distance, intra-board communication.
Common speeds are 100kbps standard mode and 400kbps fast mode.
Only two communication lines for all devices on the bus.

You can have multiple masters and multiple slaves.

Start condition is when the master node leaves the SCL high and
pulls SDA low.

If two masters want to transmit at the same time, the
first master to pull the SDA low wins the arbitration.



\subsection{Error detection}

During transmission digital signals may be corrupted due
to noise. Error detection adds additional data to a given digital
message to help detect if any error has occured during the transmission.


\textbf{Parity}

Error detecting method. An extra data bit is sent with each data
character, arranged so that the number of 1 bits in each character,
including the parity bit, is always odd or even. This can be implemented
using an XOR gate.

\textbf{Checksum}

The data is divided into k segments each of m bits.
On the sender's end, the segments are added to get the sum.
The sum is complemented to get the checksum, which is sent.
At the receiver's end all received segments are added to get the sum.
The sum is then complemented and if it is zero, the data is correct.

\textbf{Cyclic Redundancy Check (CRC)}

Based on binary division.
The cyclic redundancy check bits, are added to the end
of the data unit.
The CRC bits are the remainder of the division in which the divisor
is prechosen.
At the receiver the data is divided over the same divisor.

If a zero is caught at the receiver, the data is correct.

Polynomial generators based on the divisor value are used to generate
the CRC bits. In HDL, shift registers with XORs are used to build
the CRC circuit.

