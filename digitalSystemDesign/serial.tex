\section{UART}

Serial port is a serial communication phystical interface
through which information transfers in or out one bit at a time.
Data transfer through serial ports connected the computer
to devices such as terminals and variuos peripherals.

\textbf{Baudrate}

It is necessary to choose a proper oscillator to get the
correct baud rate with little or no error. Therefore, some common buad
rates are known. E.g. 9600.

LSB is sent first known as little-endian. Also possible but
rarely used is big-endian or MSB first.

\textbf{Parity}

Error detecting method. An extra data bit is sent with each data
character, arranged so that the number of 1 bits in each character,
including the parity bit, is always odd or even.

When setting up UART the clock frequency of the FPGA should be 16 times faster
than the baud rate.


\textbf{SPI}

Serial Peripheral Interface is a synchronous serial communication interface
for short-distance communication.
Full duplex, single master.



\textbf{Inter IC}

Serial synchronous bus. Widely used to connect low-speed ICs processing
units in short-distance, intra-board communication.
Common speeds are 100kbps standard mode and 400kbps fast mode.
Only two communication lines for all devices on the bus.

You can have multiple masters and multiple slaves.

Start condition is when the master node leaves the SCL high and
pulls SDA low.

If two masters want to transmit at the same time, the
first master to pull the SDA low wins the arbitration.
